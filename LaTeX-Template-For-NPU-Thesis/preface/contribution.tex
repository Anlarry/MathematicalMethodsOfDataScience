\renewcommand{\baselinestretch}{1.5}
\fontsize{12pt}{13pt}\selectfont
\phantomsection
\chapter*{贡~~~~献}
\addcontentsline{toc}{chapter}{\fHei 贡献}

\begin{description}
    \item[王嘉利,黄俊淞,彭冠文] 阅读以及分析题目相关信息。
    \item[黄俊淞,王嘉利,彭冠文] 提取题目核心要素并对完成题目的流程进行了讨论与规划。
    \item[王嘉利,彭冠文,黄俊淞] 得出了需要用到的技术列表。
    \item[彭冠文,王嘉利,黄俊淞] 查阅了相关文献并进行了学习研究。
    \item[王嘉利,黄俊淞] 实现了对文本的分词,清洗文件中出现的符号以及停用词并且完成了词形还原 
    \item[王嘉利,彭冠文,黄俊淞] 基于分词结果实现了TextRank和TF-IDF完成了关键词提取并对关键词进行打分
    \item[黄俊淞,彭冠文] 根据关键词提取的相关结果对文本进行向量化 
    \item[王嘉利,彭冠文] 基于sklearn K-means完成了文本聚类并于星级评分进行相关分析
    \item[黄俊淞,彭冠文] 基于ARIMA模型对评论和产品的销售情况进行了时间序列的预测和分析
    \item[彭冠文] 完成了论文的基本模板构建
    \item[王嘉利] 完成了论文第一章撰写
    \item[王嘉利,黄俊淞,彭冠文] 完成了论文第二章撰写
    \item[彭冠文] 完成了论文第三章撰写
    \item[黄俊淞] 完成了论文第四章撰写
    \item[王嘉利] 完成了论文摘要的撰写
    \item[王嘉利,黄俊淞] 完成了论文参考文献和附录的收集整理            
\end{description}

\clearpage
\endinput
