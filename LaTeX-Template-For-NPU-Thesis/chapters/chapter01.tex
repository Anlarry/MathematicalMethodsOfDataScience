\chapter{背景}

\chapter{方法}
基于所提供的数据集,我们将每一条信息分为如下的维度:

\begin{description}
    \item [文本] 每一条评论的文字中都包含了丰富的语义信息。虽然文本信息是非结构
        化,非数字化的,但现在数据科学领域内有着大量成熟的{\fKai 自然语言处
        理}技术,通过一系列算法的变换后,可以得到包含语义信息的数值化数据。
    \item [时间] 商品的声誉并非是静态的,而是随着时间变化的。数据集中给出了每一
        条评论的时间戳,通过结合其他数据维度,可以获得数据之间的时序关联。
    \item [评分] 评分是用户对产品评价最基本的数值度量,平均评分也是最为容易取得
        的可靠指标。
    \item [评价数] 出于合理的推断,我们假设一个商品的评价量$C$总是占总销售量$S$的一
        个固定比例,即满足$C/S=\textrm{const}$。又因为销售量和产品营收成直接的正
        相关关系,所以评价数也可以作为一个重要的指标。
\end{description}

在经过基础的数据处理后,我们对不同维度的指标做交叉分析。基本的流程框图
如\autoref{fig:schedule} 所示。

\begin{figure}
    \centering
    \includegraphics[height=0.3\paperwidth]{figure1.png}
    \caption{流程}
    \label{fig:schedule}
\end{figure}


\section{文本处理}

\begin{figure}
    \centering
    \includegraphics[height=0.3\paperwidth]{figure1.png}
    \caption{文本处理流程}
    \label{fig:text-schedule}
\end{figure}

本文拟采用分词,关键字提取,文本向量化等方法对文本进行处理,以得到便于计算的向量
值。基本流程图见\autoref{fig:text-schedule}。

\subsection{文本清洗}

\subsection{分词}
数据集中存在的文本是英文文本,可以。

\subsection{关键词提取}
常见的关键词提取方法有……

\subsection{文本向量化}

\section{评分与文本的关联聚类}

\chapter{结果}

\chapter{总结}

\endinput

